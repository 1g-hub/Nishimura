%\documentstyle[epsf,twocolumn]{jarticle}       %LaTeX2.09仕様
%\documentclass[twocolumn]{jarticle}     %pLaTeX2e仕様
\documentclass{jarticle}     %pLaTeX2e仕様

%一枚組だったら[twocolumn]関係のとこ消す

\setlength{\topmargin}{-45pt}
%\setlength{\oddsidemargin}{0cm} 
\setlength{\oddsidemargin}{-7.5mm}
%\setlength{\evensidemargin}{0cm} 
\setlength{\textheight}{24.1cm}
%setlength{\textheight}{25cm} 
\setlength{\textwidth}{17.4cm}
%\setlength{\textwidth}{172mm} 
\setlength{\columnsep}{11mm}

\kanjiskip=.07zw plus.5pt minus.5pt

\usepackage{graphicx}
\usepackage[dvipdfmx]{color}
\usepackage{subcaption}
\usepackage{enumerate}
\usepackage{comment}
\usepackage{url}
\usepackage{multirow}
\usepackage{diagbox}
\usepackage{algorithmic}
\usepackage{amsmath}
\usepackage{algorithm}
\usepackage{lipsum}
\usepackage[jis2004]{otf}
\usepackage{diagbox}

\begin{document}
  \noindent
  \onecolumn
  \hspace{1em}

  \today
  \hfill
  \ \  西村昭賢 

  \vspace{2mm}
  \hrule
  \begin{center}
  {\Large \bf 進捗報告}
  \end{center}
  \hrule
  \vspace{3mm}


\section{今週やったこと}
\begin{itemize}
  \item 学習済みモデルを先攻後攻両方に配置して対戦させる機能実装
  \item アグロとコントロールで 50 \% 近い勝率を記録するように対戦相手 AI 改良
  \item 数値実験
\end{itemize}

\section{学習済みモデルを先攻後攻両方に配置して対戦させる機能の実装}
今までは, 学習済みモデル vs 予め作成しておいた対戦相手しか検証ができなかったが, 学習済みモデルを先攻後攻両方に配置して対戦を実行できるようにした.


\section{アグロとコントロールで勝率調整}
先週は, アグロとコントロールの対戦相手を作成し DQN の対戦相手として適用したが, そもそもお互いの戦略同士の強さが大きく偏っていた.
そのため, 今週はなるべく 50 \% に近づくように戦略を調整した.
また, デッキが同じのままで調整することが難しかったためデッキを戦略ごとに分けてデッキ, 戦略間の勝率を調整した.
表 \ref{winrate} に調整した結果を示す. また, 表 \ref{table:agurodeck}, \ref{table:controlldeck} に各戦略におけるデッキを示す. 

\begin{table}[ht]
  \centering
  \caption{10000 回のゲーム実行における先攻の勝率}
  \label{winrate}
  \begin{tabular}{|c|c|c|}
  \hline
  \diagbox[]{先攻}{後攻}      & アグロ    & コントロール \\ \hline
  アグロ    & 0.5255 & 0.5424 \\ \hline
  コントロール & 0.5121 & 0.5053 \\ \hline
  \end{tabular}
  \end{table}

  \begin{table}[h]
    \centering
    \caption{アグロのときのデッキ}
    \label{table:agurodeck}
    \begin{tabular}{|c|c|c|c|c|c|}

    \hline
    ID & 攻撃力 & HP & コスト & 特殊効果 & 枚数 \\ \hline
    0 & 1 & 1 & 3 & 無し & 2 \\ \hline
    1 & 1 & 1 & 5 & 無し & 2 \\ \hline
    2 & 3 & 2 & 4 & 無し & 2 \\ \hline
    3 & 2 & 2 & 4 & 無し & 2 \\ \hline
    4 & 1 & 2 & 5 & 無し & 2 \\ \hline
    5 & 1 & 2 & 4 & 召喚 & 2 \\ \hline
    6 & 1 & 2 & 4 & 召喚 & 2 \\ \hline
    7 & 1 & 1 & 4 & 取得 & 2 \\ \hline
    8 & 2 & 5 & 1 & 取得 & 2 \\ \hline
    9 & 4 & 4 & 1 & 速攻 & 2 \\ \hline
    10 & 1 & 1 & 4 & 速攻 & 2 \\ \hline
    11 & 1 & 2 & 3 & 攻撃 & 2 \\ \hline
    12 & 1 & 3 & 5 & 攻撃 & 2 \\ \hline
    13 & 1 & 4 & 1 & 治癒 & 2 \\ \hline
    14 & 1 & 2 & 3 & 治癒 & 2 \\ \hline
    \end{tabular}
    \end{table}
  
    \begin{table}[h]
      \centering
      \caption{コントロールのときのデッキ}
      \label{table:controlldeck}
      \begin{tabular}{|c|c|c|c|c|c|}
      \hline
      ID & 攻撃力 & HP & コスト & 特殊効果 & 枚数 \\ \hline
      0 & 1 & 2 & 2 & 無し & 2 \\ \hline
      1 & 1 & 3 & 2 & 無し & 2 \\ \hline
      2 & 1 & 2 & 2 & 無し & 2 \\ \hline
      3 & 2 & 2 & 4 & 無し & 2 \\ \hline
      4 & 1 & 4 & 2 & 無し & 2 \\ \hline
      5 & 1 & 1 & 2 & 召喚 & 2 \\ \hline
      6 & 1 & 1 & 3 & 召喚 & 2 \\ \hline
      7 & 1 & 2 & 2 & 取得 & 2 \\ \hline
      8 & 1 & 3 & 3 & 取得 & 2 \\ \hline
      9 & 5 & 5 & 1 & 速攻 & 2 \\ \hline
      10 & 1 & 1 & 2 & 速攻 & 2 \\ \hline
      11 & 1 & 2 & 2 & 攻撃 & 2 \\ \hline
      12 & 1 & 1 & 2 & 攻撃 & 2 \\ \hline
      13 & 2& 2 & 1 & 治癒 & 2 \\ \hline
      14 & 2 & 2 & 2 & 治癒 & 2 \\ \hline
      \end{tabular}
      \end{table}


\section{数値実験}

\subsection{対戦相手を変更して DQN を適用 }
3 節で示した敵 AI を対戦相手として先攻において, DQN を用いて \textbf{後攻側の行動} を学習した. 学習後, そのまま 10000 回ゲームを実行し勝率を計算した. また, 表 \ref{table:OPdeck} に示したデッキを学習側のプレイヤーに持たせている.
勝率に有意な差が見られなかったため, 10000 回の対戦における学習済みエージェントの行動を計測し分析した. 表 \ref{action} に結果を示す. 全て同じようになった. 対戦相手の戦略に応じてではなく学習するデッキにおける戦略の最適解を学習しているのではないかと考えられる.


\begin{table}[ht]
  \centering 
  \caption{学習後 10000 回のゲーム実行における学習側の勝率 (学習側は後攻)}
  \label{winrate_after}
  \begin{tabular}{|c|c|c|c|}
    
  \hline
  \diagbox[]{勝率計算時の対戦相手}{学習時の対戦相手}    & ランダムに戦略が変化する & コントロール  & アグロ              \\ \hline
  コントロール & 0.7163      & 0.7214 & 0.7131   \\ \hline
  アグロ    & 0.7110      & 0.7108 & 0.7230       \\ \hline
  \end{tabular}
  \end{table}

  \begin{table}[h]
    \centering
    \caption{実験で用いた強すぎるカードと弱すぎるカードを含んだデッキ (ID 0 が強い, ID 14 が弱い)}
    \label{table:OPdeck}
    \begin{tabular}{|c|c|c|c|c|c|}
    \hline
    ID & 攻撃力 & HP & コスト & 特殊効果 & 枚数 \\ \hline
    0 & 4 & 4 & 1 & 無し & 2 \\ \hline
    1 & 2 & 2 & 2 & 無し & 2 \\ \hline
    2 & 3 & 3 & 3 & 無し & 2 \\ \hline
    3 & 4 & 3 & 4 & 無し & 2 \\ \hline
    4 & 5 & 4 & 5 & 無し & 2 \\ \hline
    5 & 2 & 2 & 2 & 召喚 & 2 \\ \hline
    6 & 2 & 3 & 3 & 召喚 & 2 \\ \hline
    7 & 1 & 1 & 1 & 取得 & 2 \\ \hline
    8 & 1 & 3 & 2 & 取得 & 2 \\ \hline
    9 & 2 & 1 & 2 & 速攻 & 2 \\ \hline
    10 & 3 & 1 & 3 & 速攻 & 2 \\ \hline
    11 & 1 & 2 & 2 & 攻撃 & 2 \\ \hline
    12 & 2 & 3 & 3 & 攻撃 & 2 \\ \hline
    13 & 1 & 1 & 1 & 治癒 & 2 \\ \hline
    14 & 1 & 1 & 5 & 治癒 & 2 \\ \hline
    \end{tabular}
    \end{table}


    \begin{table}[ht]
      \centering
      \caption{3 種類の学習済みエージェントの行動}
      \label{action}
      \scalebox{0.9}[0.9]{
        \begin{tabular}{|cc|cc|cc|}
          \hline
          \multicolumn{2}{|c|}{ランダムに戦略が変化}                & \multicolumn{2}{c|}{コントロール}                    & \multicolumn{2}{c|}{アグロ}                       \\ \hline
          \multicolumn{1}{|c|}{行動説明}             & 総数     & \multicolumn{1}{c|}{行動説明}             & 総数     & \multicolumn{1}{c|}{行動説明}             & 総数     \\ \hline
          \multicolumn{1}{|c|}{ターンエンド}           & 49030 & \multicolumn{1}{c|}{ターンエンド}           & 49463 & \multicolumn{1}{c|}{ターンエンド}           & 49016 \\ \hline
          \multicolumn{1}{|c|}{手札 1 を盤面に出す}      & 42920 & \multicolumn{1}{c|}{手札 1 を盤面に出す}      & 43340 & \multicolumn{1}{c|}{手札 1 を盤面に出す}      & 42950 \\ \hline
          \multicolumn{1}{|c|}{盤面 1 で相手プレイヤーに攻撃} & 39378 & \multicolumn{1}{c|}{盤面 1 で相手プレイヤーに攻撃} & 39791 & \multicolumn{1}{c|}{盤面 1 で相手プレイヤーに攻撃} & 39319 \\ \hline
          \multicolumn{1}{|c|}{盤面 2 で相手プレイヤーに攻撃} & 20684 & \multicolumn{1}{l|}{盤面 2 で相手プレイヤーに攻撃} & 20762 & \multicolumn{1}{c|}{盤面 2 で相手プレイヤーに攻撃} & 20914 \\ \hline
          \multicolumn{1}{|c|}{手札 2 を盤面に出す}      & 13673  & \multicolumn{1}{c|}{手札 2 を盤面に出す}      & 13569  & \multicolumn{1}{c|}{手札 2 を盤面に出す}      & 13337  \\ \hline
          \end{tabular}
      }
        \end{table}
\subsection{カードを 1 種類づつデッキから抜いた場合の勝率計算}
デッキ内のカードパワーを測る指標として, これまで Win Rate when Play (WRP) といった指標を試してきたが, これらの指標ではエージェントの対戦記録から計算しようとすると構築された戦略がアグロによっているためカードのコストに結果が影響されてしまう問題があった. そこで, 森先生から先週アドバイスを頂いたように, 同戦略同デッキでカードを 1 種類づつデッキから抜いた場合の勝率を計算してみた. デッキは表 \ref{table:OPdeck} を用いて, 表 \ref{winrate_aguro} に作成したアグロの行動ルーチンを先攻後攻両方に配置しカードを 1 種類ずつ抜いた場合の 10000 回ゲームを実行した際の勝率を示す.
行に注目すると, 先攻がカード ID 0 のカードをデッキから除いた時に勝率が大きく減少していることが分かる. また, カード ID 1 のカードをデッキから除いた場合に勝率が大きく増加している.
カードバワーを示す指標としてこの表は妥当性があることが分かった. 
\begin{table}[ht]
  \centering
  \caption{カードを 1 種類ずつ除いたときの 10000 回のゲーム実行における先攻の勝率}
  \label{winrate_aguro}
  \scalebox{0.80}[0.85]{
    \begin{tabular}{|c|c|c|c|c|c|c|c|c|c|c|c|c|c|c|c|}
      \cline{1-16}
      \diagbox[]{先攻}{後攻}                         & 0      & 1      & 2      & 3      & 4      & 5      & 6      & 7      & 8      & 9      & 10     & 11     & 12     & 13     & 14     \\ \hline
      \multicolumn{1}{|c|}{0}  & 0.7857 & 0.4998 & 0.4957 & 0.4779 & 0.4671 & 0.5059 & 0.4914 & 0.5153 & 0.4845 & 0.5105 & 0.5099 & 0.4953 & 0.5128 & 0.5288 & 0.4335 \\ \hline
      \multicolumn{1}{|c|}{1}  & 0.8955 & 0.6762 & 0.6798 & 0.6688 & 0.6613 & 0.6963 & 0.6789 & 0.6913 & 0.6807 & 0.6925 & 0.6914 & 0.6855 & 0.6998 & 0.6974 & 0.6345 \\ \hline
      \multicolumn{1}{|c|}{2}  & 0.8936 & 0.6763 & 0.6865 & 0.6667 & 0.6652 & 0.6961 & 0.6826 & 0.6850 & 0.6697 & 0.6839 & 0.6864 & 0.6938 & 0.6903 & 0.6958 & 0.6473 \\ \hline
      \multicolumn{1}{|c|}{3}  & 0.9088 & 0.6989 & 0.6939 & 0.6844 & 0.6700 & 0.7119 & 0.6991 & 0.6967 & 0.6932 & 0.7130 & 0.6996 & 0.7061 & 0.7116 & 0.7135 & 0.6599 \\ \hline
      \multicolumn{1}{|c|}{4}  & 0.9049 & 0.7043 & 0.7073 & 0.6816 & 0.6834 & 0.7261 & 0.7109 & 0.709  & 0.7008 & 0.7193 & 0.7173 & 0.7154 & 0.7163 & 0.7269 & 0.6729 \\ \hline
      \multicolumn{1}{|c|}{5}  & 0.8773 & 0.6547 & 0.6635 & 0.6511 & 0.6219 & 0.6721 & 0.6586 & 0.6594 & 0.6451 & 0.6780 & 0.6711 & 0.6682 & 0.6732 & 0.6748 & 0.6211 \\ \hline
      \multicolumn{1}{|c|}{6}  & 0.8929 & 0.6799 & 0.6845 & 0.6701 & 0.6673 & 0.7083 & 0.6795 & 0.6865 & 0.6802 & 0.6988 & 0.6844 & 0.7040 & 0.6925 & 0.7013 & 0.6397 \\ \hline
      \multicolumn{1}{|c|}{7}  & 0.8698 & 0.6539 & 0.6565 & 0.6439 & 0.6317 & 0.6659 & 0.6464 & 0.6602 & 0.6459 & 0.6660 & 0.6637 & 0.6626 & 0.6711 & 0.6750 & 0.6175 \\ \hline
      \multicolumn{1}{|c|}{8}  & 0.9028 & 0.6904 & 0.6897 & 0.6741 & 0.6772 & 0.7063 & 0.7011 & 0.6949 & 0.6793 & 0.7102 & 0.7046 & 0.6954 & 0.7156 & 0.7127 & 0.6621 \\ \hline
      \multicolumn{1}{|c|}{9}  & 0.8762 & 0.6568 & 0.6625 & 0.6549 & 0.6325 & 0.6677 & 0.6630 & 0.6633 & 0.6539 & 0.6782 & 0.6676 & 0.6618 & 0.6792 & 0.6737 & 0.6215 \\ \hline
      \multicolumn{1}{|c|}{10} & 0.8846 & 0.6711 & 0.6716 & 0.6443 & 0.6454 & 0.6883 & 0.6706 & 0.6724 & 0.6609 & 0.6807 & 0.6740 & 0.6744 & 0.6904 & 0.6886 & 0.6242 \\ \hline
      \multicolumn{1}{|c|}{11} & 0.8853 & 0.6671 & 0.6820 & 0.6580 & 0.6484 & 0.6960 & 0.6718 & 0.6843 & 0.6759 & 0.6840 & 0.6797 & 0.6812 & 0.6913 & 0.6862 & 0.6253 \\ \hline
      \multicolumn{1}{|c|}{12} & 0.8738 & 0.6525 & 0.6594 & 0.6407 & 0.6337 & 0.6794 & 0.6553 & 0.6651 & 0.6576 & 0.6708 & 0.6695 & 0.6689 & 0.6729 & 0.6801 & 0.6234 \\ \hline
      \multicolumn{1}{|c|}{13} & 0.8714 & 0.6428 & 0.6570 & 0.6338 & 0.6262 & 0.6751 & 0.6508 & 0.6673 & 0.6456 & 0.6666 & 0.6616 & 0.6673 & 0.6753 & 0.6710 & 0.6153 \\ \hline
      \multicolumn{1}{|c|}{14} & 0.9296 & 0.7308 & 0.7203 & 0.7199 & 0.6943 & 0.7450 & 0.7287 & 0.7391 & 0.7261 & 0.7400 & 0.7381 & 0.7334 & 0.7456 & 0.7422 & 0.6889 \\ \hline
      \end{tabular}
  }
  
  \end{table}



%index.bibはtexファイルと同階層に置く
%ちゃんと\citeしないと表示されない(1敗)
\bibliography{index.bib}
\bibliographystyle{junsrt}

\end{document}