%\documentstyle[epsf,twocolumn]{jarticle}       %LaTeX2.09仕様
%\documentclass[twocolumn]{jarticle}     %pLaTeX2e仕様
\documentclass{jarticle}     %pLaTeX2e仕様

%一枚組だったら[twocolumn]関係のとこ消す

\setlength{\topmargin}{-45pt}
%\setlength{\oddsidemargin}{0cm} 
\setlength{\oddsidemargin}{-7.5mm}
%\setlength{\evensidemargin}{0cm} 
\setlength{\textheight}{24.1cm}
%setlength{\textheight}{25cm} 
\setlength{\textwidth}{17.4cm}
%\setlength{\textwidth}{172mm} 
\setlength{\columnsep}{11mm}

\kanjiskip=.07zw plus.5pt minus.5pt

\usepackage{graphicx}
\usepackage[dvipdfmx]{color}
\usepackage{subcaption}
\usepackage{enumerate}
\usepackage{comment}
\usepackage{url}
\usepackage{multirow}
\usepackage{diagbox}
\usepackage{algorithmic}
\usepackage{amsmath}
\usepackage{algorithm}
\usepackage{lipsum}
\usepackage[jis2004]{otf}


\begin{document}
  \noindent
  \onecolumn
  \hspace{1em}

  \today
  \hfill
  \ \ B3 西村昭賢 

  \vspace{2mm}
  \hrule
  \begin{center}
  {\Large \bf 進捗報告}
  \end{center}
  \hrule
  \vspace{3mm}


\section{今週やったこと}
\begin{itemize}
  \item B3 実験用データ集め, 資料とスライド作成
  \item カードの特殊効果実装
  \item 初期手札を変えた際の勝率の変化
  \item ゲームバランス調整手法の検討
\end{itemize}

\section{カードの特殊効果実装}
環境に変化を加える時のために, 以下の 2 つの特殊効果を実装した.
\begin{itemize}
  \item 自身が盤面にいる時は対戦相手は自身しか攻撃ができない 
  \item 盤面に出す時, 自盤面と敵盤面のカードを全て破壊する
\end{itemize}
デバックは終わっているため入れようと思えばいつでもデッキに入れることができるが, まずは現在の環境で成果を出してからにする.

\section{初期手札を変えたときの勝率の変化}
先週にランダム環境における後攻の初期手札の枚数を 3 → 4 枚に変更しただけで後攻の勝率が 1 割近く向上した. この結果を受けて, デッキのバランスの悪さを表す指標になるのではないかと考えていた. 森先生に初期手札の組をいろいろ変えてやってみるようアドバイスを頂いたので実験した. 表 \ref{table:deck} にデッキの内容を示す. 変更点は無い.


\begin{table}[h]
  \centering
  \caption{デッキの内容}
  \label{table:deck}
  \begin{tabular}{|c|c|c|c|c|c|}
  \hline
  ID & 攻撃力 & HP & コスト & 特殊効果 & 枚数 \\ \hline
  0 & 1 & 1 & 0 & 無し & 2 \\ \hline
  1 & 2 & 1 & 1 & 無し & 2 \\ \hline
  2 & 3 & 2 & 2 & 無し & 2 \\ \hline
  3 & 4 & 3 & 3 & 無し & 2 \\ \hline
  4 & 5 & 4 & 4 & 無し & 2 \\ \hline
  5 & 2 & 2 & 2 & 召喚 & 2 \\ \hline
  6 & 2 & 3 & 3 & 召喚 & 2 \\ \hline
  7 & 1 & 1 & 1 & 循環 & 2 \\ \hline
  8 & 1 & 3 & 2 & 循環 & 2 \\ \hline
  9 & 2 & 1 & 2 & 速攻 & 2 \\ \hline
  10 & 3 & 1 & 3 & 速攻 & 2 \\ \hline
  11 & 1 & 2 & 2 & 攻撃 & 2 \\ \hline
  12 & 2 & 3 & 3 & 攻撃 & 2 \\ \hline
  13 & 1 & 1 & 1 & 治癒 & 2 \\ \hline
  14 & 2 & 1 & 3 & 治癒 & 2 \\ \hline
  \end{tabular}
  \end{table}

  カードの特殊効果は, 
  \begin{quote}
    \begin{itemize}
     \item 盤面に出したら (攻撃力 , HP) = ( 1 , 1 )のユニット追加で出す. (召喚)
     \item 盤面に出したら自プレイヤーの HP を 2 回復 (治癒)
     \item 盤面に出したら敵プレイヤーの HP を 2 削る (攻撃)
     \item 盤面に出したら自プレイヤーは 1 枚カードをドロー (循環)
     \item 盤面に出たターンに攻撃できる (速攻)
    \end{itemize}
   \end{quote}
   となっている.
  なお, 結果として示しやすいようにカードに ID を割り振っている.

  先週にカードパワーを図る指標として導入した Win Rate when Play (WRP) が高い順から 5 分割してそれぞれ 3 種類ずつ後攻の初期手札として決定してランダム動詞の対戦環境の後攻の勝率を計算した. 表 \ref{changeinihand} に結果を示す.
  \begin{table}[ht]
    \caption{初期手札の組合せを変化させたときの勝率}
    \centering
    \label{changeinihand}
    \begin{tabular}{|c|l|}
    \hline
    \begin{tabular}[c]{@{}c@{}}初期手札\\ カード ID\end{tabular} & \multicolumn{1}{c|}{勝率} \\ \hline
    4, 3, 6                                               & 0.57672                 \\ \hline
    8, 12, 5                                              & 0.57734                 \\ \hline
    2, 7, 11                                              & 0.51678                 \\ \hline
    1, 14, 10                                             & 0.37751                 \\ \hline
    0, 13, 9                                              & 0.37728                 \\ \hline
    \end{tabular}
  \end{table}
  初期ドローによる勝率の変化が大きかったため, 新しく Win Rate when Draw (WRD) の指標を追加して確認してみた. 表 \ref{table:WRP} に結果を示す.

  \begin{table}[ht]
    \centering
    \caption{カードごとの WRD (降順)}
    \vspace{-0.3cm}
    \label{table:WRP}
    \scalebox{1.0}[1.0]{
      \begin{tabular}{|c|c|}
        \hline
        カード ID  & WRD \\ \hline
        5 & 0.51898 \\ \hline
        4 & 0.51643 \\ \hline
        8 & 0.51609 \\ \hline
        3 & 0.51453 \\ \hline
        7 & 0.51439 \\ \hline
        6 & 0.51276 \\ \hline
        2 & 0.51272 \\ \hline
        12 & 0.51010 \\ \hline
        13 & 0.50555 \\ \hline
        11 & 0.50461 \\ \hline
        1 & 0.50437 \\ \hline
        0 & 0.50294 \\ \hline
        10 & 0.50016 \\ \hline
        9 & 0.50007 \\ \hline
        14 & 0.49625 \\ \hline
        \end{tabular}
    }
    \end{table}

\section{考えてみたこと}
\begin{enumerate}
  \item 深層強化学習でバランス調整したいデッキを用いてアグロ, コントロール寄りの戦略を取る対戦相手に対してどちらにも勝てる戦略を作成する.
  \item 作成したエージェントの戦略下で, WRP や WRD などを用いてカードパワーを調べる.
  \item バフ・ナーフするカードを限定し, 目標となる勝率になるまで GA を用いてカードのパラメータを調整する.
\end{enumerate}

\subsection{GA でパラメータ調整}
カードの HP, 攻撃力, コストのどこをどう調節するかということが定式化できればよかったが, そこで行き詰まっているため, カードの 3 種のパラメータの調整を, 組合せ最適化問題, 離散最適化問題とみなして遺伝的アルゴリズムを用いて調節する方法をとった.
なお, GA の実装の都合上デッキの初期値を表 \ref{table:Redeck} のように変更した. また, DQN の適用実験についてプレイヤーの初期手札を 5 枚と変更していたため, ランダム同士の対戦環境でも初期手札の枚数を 5 枚とした.

\begin{table}[h]
  \centering
  \caption{GA の実装の都合上変更したデッキの内容}
  \label{table:Redeck}
  \begin{tabular}{|c|c|c|c|c|c|}
  \hline
  ID & 攻撃力 & HP & コスト & 特殊効果 & 枚数 \\ \hline
  0 & 1 & 2 & 1 & 無し & 2 \\ \hline
  1 & 2 & 2 & 2 & 無し & 2 \\ \hline
  2 & 3 & 3 & 3 & 無し & 2 \\ \hline
  3 & 4 & 3 & 4 & 無し & 2 \\ \hline
  4 & 5 & 4 & 5 & 無し & 2 \\ \hline
  5 & 2 & 2 & 2 & 召喚 & 2 \\ \hline
  6 & 2 & 3 & 3 & 召喚 & 2 \\ \hline
  7 & 1 & 1 & 1 & 循環 & 2 \\ \hline
  8 & 1 & 3 & 2 & 循環 & 2 \\ \hline
  9 & 2 & 1 & 2 & 速攻 & 2 \\ \hline
  10 & 3 & 1 & 3 & 速攻 & 2 \\ \hline
  11 & 1 & 2 & 2 & 攻撃 & 2 \\ \hline
  12 & 2 & 3 & 3 & 攻撃 & 2 \\ \hline
  13 & 1 & 1 & 1 & 治癒 & 2 \\ \hline
  14 & 2 & 1 & 3 & 治癒 & 2 \\ \hline
  \end{tabular}
  \end{table}

この時, 50000 回ランダム同士で対戦させたところ, 後攻の勝率が 0.32962 となった.
まず, 何も制限を掛けずにカードの攻撃力と HP , コストについて遺伝的アルゴリズムを用いてランダム同士の対戦環境において後攻の勝率が 0.50 に近づくように調整した. 
デッキにおいて ID 0 〜 14 のカードについてそれぞれ 3 つパラメータがあり, 計 45 個パラメータが存在する. 45 要素の一次元配列として遺伝的アルゴリズムに適用した.
個体の適応度として, 一定回数シミュレーションを行い勝率を計算し
\begin{equation}
  fitness = (0.5 - \mathrm{winrate})^2
\end{equation}
で計算した.  なお, この定義の仕方のため適応度の値が小さい個体ほど優秀な個体とされる.
表 に各種パラメータを示す.

\begin{table}[t]
  \centering
  \caption{遺伝的アルゴリズムパラメータ}
  \begin{tabular}{|c|c|}
  \hline
  パラメータ       & 値            \\ \hline \hline
  1 世代あたりの個体数 & 100          \\ \hline
  世代数         & 50           \\ \hline
  交叉の確率       & 0.4          \\ \hline
  交叉の種類       & 二点交叉         \\ \hline
  突然変異の確率     & 0.2          \\ \hline
  選択          & 3 個体のトーナメント方式 \\ \hline
  勝率計算の際の回数 & 500 \\ \hline
  \end{tabular}
  \end{table}

表 \ref{table:resultdeck} に結果として得られたデッキを示す.
\begin{table}[t]
  \centering
  \caption{GA を全てのカードに対して適用した場合のデッキの内容}
  \label{table:resultdeck}
  \begin{tabular}{|c|c|c|c|c|c|}
  \hline
  ID & 攻撃力 & HP & コスト & 特殊効果 & 枚数 \\ \hline
  0 & 1 & 2 & 1 & 無し & 2 \\ \hline
  1 & 2 & 3 & 4 & 無し & 2 \\ \hline
  2 & 1 & 3 & 3 & 無し & 2 \\ \hline
  3 & 1 & 3 & 4 & 無し & 2 \\ \hline
  4 & 1 & 4 & 5 & 無し & 2 \\ \hline
  5 & 2 & 4 & 2 & 召喚 & 2 \\ \hline
  6 & 1 & 3 & 3 & 召喚 & 2 \\ \hline
  7 & 1 & 5 & 2 & 循環 & 2 \\ \hline
  8 & 1 & 3 & 2 & 循環 & 2 \\ \hline
  9 & 2 & 1 & 2 & 速攻 & 2 \\ \hline
  10 & 3 & 1 & 3 & 速攻 & 2 \\ \hline
  11 & 1 & 2 & 3 & 攻撃 & 2 \\ \hline
  12 & 1 & 3 & 3 & 攻撃 & 2 \\ \hline
  13 & 1 & 4 & 1 & 治癒 & 2 \\ \hline
  14 & 2 & 1 & 3 & 治癒 & 2 \\ \hline
  \end{tabular}
  \end{table}
変更されたパラメータ量は合計 25 となり原型をとどめていない. 
これを深層強化学習で最適化するカードを限定させ, マナレシオといったカードゲームにおけるヒューリスティックな知識を用いて変化量を抑えることができれば良いと考えている.

\subsection{学習中に戦略が変わる相手に対しての DQN }
以前までの対戦相手のルーチンは, 相手の盤面に破壊できるカードがあれば攻撃し, 緊急時以外は相手プレイヤーを攻撃してくるものだった. 今回はAlgorithm \ref{alg1} に示すような盤面を積極的に処理するルーチンを作成し, 対戦相手のルーチンを学習のエピソードごとに乱数で変化させて学習した.
\begin{figure}[t]
  \vspace{-0.3cm}
  \begin{algorithm}[H]
      \caption{
        新たに作成した盤面処理を優先する行動ルーチン
        }
      \label{alg1}
      \begin{algorithmic}[1] 
      \STATE 盤面にカードを 1 枚プレイ
      \FOR{盤面のカード (プレイ順が古い方から)}
      \IF{敵の盤面に 1 回の攻撃で破壊できるカードがある}
      \STATE その攻撃対象を選んで攻撃
      \ELSE
      \IF{敵盤面にカードがある}
      \STATE 敵盤面の最も攻撃力高いカードを攻撃
      \ELSE
      \STATE 敵プレイヤーを攻撃
      \ENDIF
      \ENDIF
      \ENDFOR
      \STATE ターンを終了
      \end{algorithmic}
  \end{algorithm}
  \vspace{-0.3cm}
  \end{figure}
先攻で 1000000 ステップ学習し, 10000 回対戦を行った結果, 0.9806 という勝率を得た.
また, 表 \ref{table:WRPWRD} に学習済みモデルで WRP , WRD を計測した結果を示す.
上位 5 つ, 下位 3 つが同じカードとなった.

\begin{table}[t]
  \caption{学習済みモデルにおける WRP , WRD の計測結果 (降順)}
  \centering
  \label{table:WRPWRD}
  \begin{tabular}{|cc|cc|}
  \hline
  \multicolumn{2}{|c|}{WRP}          & \multicolumn{2}{c|}{WRD}          \\ \hline
  \multicolumn{1}{|c|}{ID} & 値       & \multicolumn{1}{c|}{ID} & 値       \\ \hline
  \multicolumn{1}{|c|}{8}  & 0.97787 & \multicolumn{1}{c|}{8}  & 0.97892 \\ \hline
  \multicolumn{1}{|c|}{5}  & 0.97720 & \multicolumn{1}{c|}{5}  & 0.97861 \\ \hline
  \multicolumn{1}{|c|}{7}  & 0.97631 & \multicolumn{1}{c|}{7}  & 0.97703 \\ \hline
  \multicolumn{1}{|c|}{0}  & 0.97585 & \multicolumn{1}{c|}{0}  & 0.97663 \\ \hline
  \multicolumn{1}{|c|}{1}  & 0.97418 & \multicolumn{1}{c|}{1}  & 0.97615 \\ \hline
  \multicolumn{1}{|c|}{13} & 0.97346 & \multicolumn{1}{c|}{2}  & 0.97563 \\ \hline
  \multicolumn{1}{|c|}{11} & 0.97288 & \multicolumn{1}{c|}{11} & 0.97520 \\ \hline
  \multicolumn{1}{|c|}{6}  & 0.97287 & \multicolumn{1}{c|}{4}  & 0.97510 \\ \hline
  \multicolumn{1}{|c|}{2}  & 0.97239 & \multicolumn{1}{c|}{3}  & 0.97507 \\ \hline
  \multicolumn{1}{|c|}{9}  & 0.97235 & \multicolumn{1}{c|}{6}  & 0.97496 \\ \hline
  \multicolumn{1}{|c|}{4}  & 0.97076 & \multicolumn{1}{c|}{13} & 0.97486 \\ \hline
  \multicolumn{1}{|c|}{3}  & 0.97040 & \multicolumn{1}{c|}{9}  & 0.97434 \\ \hline
  \multicolumn{1}{|c|}{12} & 0.97001 & \multicolumn{1}{c|}{12} & 0.97384 \\ \hline
  \multicolumn{1}{|c|}{10} & 0.96775 & \multicolumn{1}{c|}{10} & 0.97304 \\ \hline
  \multicolumn{1}{|c|}{14} & 0.96474 & \multicolumn{1}{c|}{14} & 0.97178 \\ \hline
  \end{tabular}
  \end{table}

\section{今後やること}
表 \ref{table:WRPWRD} から調整対象のカードを選んで, 遺伝的アルゴリズムに適用させる. どのようにカードを選ぶか, また調整したデッキの評価方法を検討する必要がある. 
%index.bibはtexファイルと同階層に置く
%ちゃんと\citeしないと表示されない(1敗)

\end{document}