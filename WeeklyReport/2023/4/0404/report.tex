%\documentstyle[epsf,twocolumn]{jarticle}       %LaTeX2.09仕様
%\documentclass[twocolumn]{jarticle}     %pLaTeX2e仕様
\documentclass{jarticle}     %pLaTeX2e仕様

%一枚組だったら[twocolumn]関係のとこ消す

\setlength{\topmargin}{-45pt}
%\setlength{\oddsidemargin}{0cm} 
\setlength{\oddsidemargin}{-7.5mm}
%\setlength{\evensidemargin}{0cm} 
\setlength{\textheight}{24.1cm}
%setlength{\textheight}{25cm} 
\setlength{\textwidth}{17.4cm}
%\setlength{\textwidth}{172mm} 
\setlength{\columnsep}{11mm}

\kanjiskip=.07zw plus.5pt minus.5pt

\usepackage{graphicx}
\usepackage[dvipdfmx]{color}
\usepackage{subcaption}
\usepackage{enumerate}
\usepackage{comment}
\usepackage{url}
\usepackage{multirow}
\usepackage{diagbox}

\begin{document}
  \noindent
  \hspace{1em}

  \today ゼミ資料
  \hfill
  \ \  西村昭賢 

  \vspace{2mm}
  \hrule
  \begin{center}
  {\Large \bf 進捗報告}
  \end{center}
  \hrule
  \vspace{3mm}

\section{したこと}
\begin{itemize}
  \item カードのパラメータを変更した場合の勝率変動調査
  \item 今後の方針
\end{itemize}

\section{カードのパラメータを変更した場合の勝率変動調査}
卒業研究では, カードを1つずつ除いてデッキ内のカードパワーを測定後, 自分が勝手に GA 使って自動調整する方向に持っていってしまった. カードパワー測定後に森先生に頂いていた「カードのパラメータ(HP, 攻撃力, コスト) を変更してみてどのように勝率が変わるか見てみると面白いかもしれない」というアドバイスを参考に, カードのパラメータを変更した場合の勝率の変動を調査した. 

\subsection{数値実験}
表 1 に用いたデッキを示す.

\begin{table}[ht]
  \centering
  \caption{強いカード, 弱いカードを 1 種類ずつ恣意的に加えたデッキ}
  \label{table:OPdeck}
  \vspace{-0.3cm}
  \scalebox{0.70}[0.70]{
    \begin{tabular}{|c|c|c|c|c|c|}
      \hline
      ID & 攻撃力 & HP & コスト & 特殊効果 & 枚数 \\ \hline
      0 & 4 & 4 & 1 & 無し & 2 \\ \hline
      1 & 2 & 2 & 2 & 無し & 2 \\ \hline
      2 & 3 & 3 & 3 & 無し & 2 \\ \hline
      3 & 4 & 3 & 4 & 無し & 2 \\ \hline
      4 & 5 & 4 & 5 & 無し & 2 \\ \hline
      5 & 2 & 2 & 2 & 召喚 & 2 \\ \hline
      6 & 2 & 3 & 3 & 召喚 & 2 \\ \hline
      7 & 1 & 1 & 1 & 取得 & 2 \\ \hline
      8 & 1 & 3 & 2 & 取得 & 2 \\ \hline
      9 & 2 & 1 & 2 & 速攻 & 2 \\ \hline
      10 & 3 & 1 & 3 & 速攻 & 2 \\ \hline
      11 & 1 & 2 & 2 & 攻撃 & 2 \\ \hline
      12 & 2 & 3 & 3 & 攻撃 & 2 \\ \hline
      13 & 1 & 1 & 1 & 治癒 & 2 \\ \hline
      14 & 1 & 1 & 5 & 治癒 & 2 \\ \hline
      \end{tabular}
  }
  
  \end{table}

  自分が構築したアグロの戦略同士で 10000 回ずつ対戦させ, 勝率を計算した.
  カードは表 \ref{table:OPdeck} の各 ID ごと 1 種類ずつパラメータを変更した. 調整するパラメータはカードの 攻撃力, HP, コストの 3 種類で各パラメータの値域は 1 $\sim$ 5 としてカード 1 種類あたり 125 通りのパラメータの組合せを実験した.

  \subsection{結果と考察}
  カードを 1 種類のみ変更しただけだったため, 大きな変化は見られなかった. 
  しかし, ID 0 のカードに関しては, 攻撃力が 3 以上の時, コストが 1 以外の時に先攻の勝率がコストが 1 の時と比較して大きく増加していた.\par
  卒業研究の方法では, 表 \ref{winrate_aguro} から ID 0 のカードがもっとも強く, ID 14 のカードが最も弱いと判断できたが, ID 14 のカードのパラメータを変更させた場合は大きな変化は見られなかった.\par
  マナコストがデッキ内の他のカードに近づくほど先攻の勝率が上がっていくことから, 変更前のカードID 0 のようにマナコストの観点から極端に強いカードがあると後攻もそのカードが使えるためその分先攻の勝率が低くなることが分かる.\par
  また, コストが 1 である時でも, 攻撃力が 3 以上の時に勝率が変動していることから攻撃力が HP より勝率に与える影響が大きいことが分かる. これはアグロ同士の対戦であるため, その戦略が反映された結果であると考えられる. \par
  このように戦略がある程度反映されるため, このような調査からなにか定性的な性質を見つけることができればより少ない工数で効果的なカードのパラメータ調整できる可能性がある.学習済みエージェント同士でも同じ実験を試してみたい.


  \begin{table}[t]
    \centering
    \caption{アグロ同士におけるカードを 1 種類ずつ除いたときの先攻の勝率}
    \label{winrate_aguro}
    \vspace{-0.3cm}
    \scalebox{0.65}[0.65]{
      \begin{tabular}{|c|c|c|c|c|c|c|c|c|c|c|c|c|c|c|c|}
        \cline{1-16}
        \diagbox[]{先攻}{後攻}                         & 0      & 1      & 2      & 3      & 4      & 5      & 6      & 7      & 8      & 9      & 10     & 11     & 12     & 13     & 14     \\ \hline
        \multicolumn{1}{|c|}{0}  & 0.7857 & 0.4998 & 0.4957 & 0.4779 & 0.4671 & 0.5059 & 0.4914 & 0.5153 & 0.4845 & 0.5105 & 0.5099 & 0.4953 & 0.5128 & 0.5288 & \textbf{0.4335} \\ \hline
        \multicolumn{1}{|c|}{1}  & 0.8955 & 0.6762 & 0.6798 & 0.6688 & 0.6613 & 0.6963 & 0.6789 & 0.6913 & 0.6807 & 0.6925 & 0.6914 & 0.6855 & 0.6998 & 0.6974 & 0.6345 \\ \hline
        \multicolumn{1}{|c|}{2}  & 0.8936 & 0.6763 & 0.6865 & 0.6667 & 0.6652 & 0.6961 & 0.6826 & 0.6850 & 0.6697 & 0.6839 & 0.6864 & 0.6938 & 0.6903 & 0.6958 & 0.6473 \\ \hline
        \multicolumn{1}{|c|}{3}  & 0.9088 & 0.6989 & 0.6939 & 0.6844 & 0.6700 & 0.7119 & 0.6991 & 0.6967 & 0.6932 & 0.7130 & 0.6996 & 0.7061 & 0.7116 & 0.7135 & 0.6599 \\ \hline
        \multicolumn{1}{|c|}{4}  & 0.9049 & 0.7043 & 0.7073 & 0.6816 & 0.6834 & 0.7261 & 0.7109 & 0.709  & 0.7008 & 0.7193 & 0.7173 & 0.7154 & 0.7163 & 0.7269 & 0.6729 \\ \hline
        \multicolumn{1}{|c|}{5}  & 0.8773 & 0.6547 & 0.6635 & 0.6511 & 0.6219 & 0.6721 & 0.6586 & 0.6594 & 0.6451 & 0.6780 & 0.6711 & 0.6682 & 0.6732 & 0.6748 & 0.6211 \\ \hline
        \multicolumn{1}{|c|}{6}  & 0.8929 & 0.6799 & 0.6845 & 0.6701 & 0.6673 & 0.7083 & 0.6795 & 0.6865 & 0.6802 & 0.6988 & 0.6844 & 0.7040 & 0.6925 & 0.7013 & 0.6397 \\ \hline
        \multicolumn{1}{|c|}{7}  & 0.8698 & 0.6539 & 0.6565 & 0.6439 & 0.6317 & 0.6659 & 0.6464 & 0.6602 & 0.6459 & 0.6660 & 0.6637 & 0.6626 & 0.6711 & 0.6750 & 0.6175 \\ \hline
        \multicolumn{1}{|c|}{8}  & 0.9028 & 0.6904 & 0.6897 & 0.6741 & 0.6772 & 0.7063 & 0.7011 & 0.6949 & 0.6793 & 0.7102 & 0.7046 & 0.6954 & 0.7156 & 0.7127 & 0.6621 \\ \hline
        \multicolumn{1}{|c|}{9}  & 0.8762 & 0.6568 & 0.6625 & 0.6549 & 0.6325 & 0.6677 & 0.6630 & 0.6633 & 0.6539 & 0.6782 & 0.6676 & 0.6618 & 0.6792 & 0.6737 & 0.6215 \\ \hline
        \multicolumn{1}{|c|}{10} & 0.8846 & 0.6711 & 0.6716 & 0.6443 & 0.6454 & 0.6883 & 0.6706 & 0.6724 & 0.6609 & 0.6807 & 0.6740 & 0.6744 & 0.6904 & 0.6886 & 0.6242 \\ \hline
        \multicolumn{1}{|c|}{11} & 0.8853 & 0.6671 & 0.6820 & 0.6580 & 0.6484 & 0.6960 & 0.6718 & 0.6843 & 0.6759 & 0.6840 & 0.6797 & 0.6812 & 0.6913 & 0.6862 & 0.6253 \\ \hline
        \multicolumn{1}{|c|}{12} & 0.8738 & 0.6525 & 0.6594 & 0.6407 & 0.6337 & 0.6794 & 0.6553 & 0.6651 & 0.6576 & 0.6708 & 0.6695 & 0.6689 & 0.6729 & 0.6801 & 0.6234 \\ \hline
        \multicolumn{1}{|c|}{13} & 0.8714 & 0.6428 & 0.6570 & 0.6338 & 0.6262 & 0.6751 & 0.6508 & 0.6673 & 0.6456 & 0.6666 & 0.6616 & 0.6673 & 0.6753 & 0.6710 & 0.6153 \\ \hline
        \multicolumn{1}{|c|}{14} & \textbf{0.9296} & 0.7308 & 0.7203 & 0.7199 & 0.6943 & 0.7450 & 0.7287 & 0.7391 & 0.7261 & 0.7400 & 0.7381 & 0.7334 & 0.7456 & 0.7422 & 0.6889 \\ \hline
        \end{tabular}
    }
      \end{table}

\section{今後の方針}
JSAI で 6 / 7 (水) に口頭発表で採択されたためそれまでは発表に備えたデータ取りを中心に今の研究を進めていく方針で考えています.  
\par
それ以降はアクションゲームとか対象とした研究とかに挑戦してみたい気持ちもありますが, まだ具体的に先生に提案できるテーマは決まっていません.


%index.bibはtexファイルと同階層に置く
%ちゃんと\citeしないと表示されない(1敗)
\bibliography{index.bib}
\bibliographystyle{junsrt}

\end{document}