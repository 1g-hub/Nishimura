%\documentstyle[epsf,twocolumn]{jarticle}       %LaTeX2.09仕様
%\documentclass[twocolumn]{jarticle}     %pLaTeX2e仕様
\documentclass{jarticle}     %pLaTeX2e仕様

%一枚組だったら[twocolumn]関係のとこ消す

\setlength{\topmargin}{-45pt}
%\setlength{\oddsidemargin}{0cm} 
\setlength{\oddsidemargin}{-7.5mm}
%\setlength{\evensidemargin}{0cm} 
\setlength{\textheight}{24.1cm}
%setlength{\textheight}{25cm} 
\setlength{\textwidth}{17.4cm}
%\setlength{\textwidth}{172mm} 
\setlength{\columnsep}{11mm}

\kanjiskip=.07zw plus.5pt minus.5pt

\usepackage{graphicx}
\usepackage[dvipdfmx]{color}
\usepackage{subcaption}
\usepackage{enumerate}
\usepackage{comment}
\usepackage{url}
\usepackage{multirow}
\usepackage{diagbox}


\begin{document}
  \noindent
  \onecolumn
  \hspace{1em}

  \today
  ゼミ資料
  \hfill
  \ \ B3 西村昭賢 

  \vspace{2mm}
  \hrule
  \begin{center}
  {\Large \bf 進捗報告}
  \end{center}
  \hrule
  \vspace{3mm}


\section{今日やったこと}

\begin{quote}
  \begin{itemize}
   \item pythonでカードゲームの再実装
   \item 実験環境の構築
  \end{itemize}
 \end{quote}

\section{Pythonで実装したカードゲーム}
 Python だけでカードゲームの対戦を行えるように, Python で簡単なカードゲームを作成した.
 まだルールははっきりとは決めてはいないが,プレイヤーはHPを持ち,カードは攻撃力とHPをそれぞれ持つ.
 またカードが敵カードに攻撃する際には,敵カードの攻撃力分カードがダメージを受けるものとした.
 カードのコスト,それに関するマナコスト,カードに付随する特殊効果などは未実装である.

\section{実験環境の構築}


\section{???}


%index.bibはtexファイルと同階層に置く
%ちゃんと\citeしないと表示されない(1敗)
\bibliography{index.bib}
\bibliographystyle{junsrt}

\end{document}