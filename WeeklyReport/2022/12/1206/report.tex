
%\documentstyle[epsf,twocolumn]{jarticle}       %LaTeX2.09仕様
%\documentclass[twocolumn]{jarticle}     %pLaTeX2e仕様
\documentclass{jarticle}     %pLaTeX2e仕様

%一枚組だったら[twocolumn]関係のとこ消す

\setlength{\topmargin}{-45pt}
%\setlength{\oddsidemargin}{0cm} 
\setlength{\oddsidemargin}{-7.5mm}
%\setlength{\evensidemargin}{0cm} 
\setlength{\textheight}{24.1cm}
%setlength{\textheight}{25cm} 
\setlength{\textwidth}{17.4cm}
%\setlength{\textwidth}{172mm} 
\setlength{\columnsep}{11mm}

\kanjiskip=.07zw plus.5pt minus.5pt

\usepackage{graphicx}
\usepackage[dvipdfmx]{color}
\usepackage{subcaption}
\usepackage{enumerate}
\usepackage{comment}
\usepackage{url}
\usepackage{multirow}
\usepackage{diagbox}


\begin{document}
  \noindent
  \onecolumn
  \hspace{1em}

  \today
  \hfill
  \ \ B3 西村昭賢 

  \vspace{2mm}
  \hrule
  \begin{center}
  {\Large \bf 進捗報告}
  \end{center}
  \hrule
  \vspace{3mm}


\section{今週やったこと}
\begin{quote}
  \begin{itemize}
   \item 学習ステップ数を増やした DQN の結果
   \item DQN の実験の結果を踏まえた改善
   \item 再実験
   \item コスト, HP ,特殊効果追加した ver の環境の作成
  \end{itemize}
 \end{quote}


\section{学習ステップを増やした DQN の実験結果}
先週, DQN で 200000 ステップ(約 5000 エピソード)学習した結果先手が勝率 5 割を切るという結果となった.一方, 新たに実装したモンテカルロ探索では 1000000 エピソード学習し先手で 0.8011 という勝率を残した.\par
この結果を受け, DQN のステップ数を増やして学習が行われるか実験を行った.パラメータは表 \ref{table:param} に示す.
\begin{table}[h]
  \centering
  \caption{DQNのパラメータ}
  \label{table:param}
  \begin{tabular}{|c||c|}
  \hline
  方策                 & ε-greedy \\ \hline
  ε                      & 0.1      \\ \hline
  全結合層の活性化関数             & ReLU     \\ \hline
  全結合層の次元                & 64       \\ \hline
  最適化アルゴリズム              & Adam     \\ \hline
  学習率                    & 1e-2     \\ \hline
  Experience Replayのメモリ量 & 1000000  \\ \hline
  \end{tabular}
  \end{table}

\section{???}


\section{???}


%index.bibはtexファイルと同階層に置く
%ちゃんと\citeしないと表示されない(1敗)
\bibliography{index.bib}
\bibliographystyle{junsrt}

\end{document}