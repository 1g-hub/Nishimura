%\documentstyle[epsf,twocolumn]{jarticle}       %LaTeX2.09仕様
%\documentclass[twocolumn]{jarticle}     %pLaTeX2e仕様
\documentclass{jarticle}     %pLaTeX2e仕様

%一枚組だったら[twocolumn]関係のとこ消す

\setlength{\topmargin}{-45pt}
%\setlength{\oddsidemargin}{0cm} 
\setlength{\oddsidemargin}{-7.5mm}
%\setlength{\evensidemargin}{0cm} 
\setlength{\textheight}{24.1cm}
%setlength{\textheight}{25cm} 
\setlength{\textwidth}{17.4cm}
%\setlength{\textwidth}{172mm} 
\setlength{\columnsep}{11mm}

\kanjiskip=.07zw plus.5pt minus.5pt

\usepackage{graphicx}
\usepackage[dvipdfmx]{color}
\usepackage{subcaption}
\usepackage{enumerate}
\usepackage{comment}
\usepackage{url}
\usepackage{multirow}
\usepackage{diagbox}
\usepackage{amsmath,amssymb}
\usepackage{mathtools}
\usepackage{wrapfig}
\usepackage{graphicx}
\usepackage{float}
\usepackage{algorithmic}
\usepackage{algorithm}


\begin{document}

  \noindent
  \onecolumn
  \hspace{1em}

  \today
  \hfill
  \ \  B3 西村昭賢 

  \vspace{2mm}
  \hrule
  \begin{center}
  {\Large \bf 進捗報告}
  \end{center}
  \hrule
  \vspace{3mm}


\section{今週やったこと}

\begin{quote}
  \begin{itemize}
   \item 研究発表会の準備(スライド, 資料)
   \par
   資料とスライドに関して修正, アドバイス頂いた方々ありがとうございました.
   \item 実際の TCG ルールに寄せた環境におけるルールベースで作成した敵に対するエージェントの学習実験
  \end{itemize}
 \end{quote}

\section{先週までの進捗とステップを増やした場合の実験}
先週は以下に示すようにプレイヤーの HP , マナコスト, カードの特殊効果など実際の TCG ルールに寄せた環境を作成し実験した.

\begin{quote}
  \begin{itemize}
   \item プレイヤー
   \par
   \begin{quote}
    \begin{itemize}
     \item HP
     \par
     最大 20, 0 となればゲーム敗北
     \item マナコスト
     \par
     ゲーム開始時1 , 最大 5 , ターンごとに 1 増加
     \item ライブラリ
     \par
     ライブラリは30枚のカードを持つ
    \end{itemize}
   \end{quote}
   \item カード
   \begin{quote}
    \begin{itemize}
     \item コスト
     \par
     盤面にプレイする際にカードのコスト分プレイヤーのマナコスト減少
     \item 特殊効果
     \par
     \begin{quote}
      \begin{itemize}
       \item 盤面に出したら (攻撃力 , HP) = ( 1 , 1 )のユニット追加で出す. (召喚)
       \item 盤面に出したら自プレイヤーの HP を 2 回復 (治癒)
       \item 盤面に出したら敵プレイヤーの HP を 2 削る (攻撃)
       \item 盤面に出したら自プレイヤーは 1 枚カードをドロー (循環)
       \item 盤面に出たターンに攻撃できる (速攻)
      \end{itemize}
     \end{quote}
    \end{itemize}
   \end{quote}
   \item 終了条件
   \par
   どちらかのプレイヤーの体力が 0 以下となった ,または デッキ切れの状態でドローしようとした時

  \end{itemize}
 \end{quote}

 \subsection{実験条件}
 お互いのライブラリは等しくした.表 \ref{table:deck} にライブラリの内容を示す.
 \begin{table}[h]
  \centering
  \caption{ライブラリの内容}
  \label{table:deck}
  \begin{tabular}{|c|c|c|c|c|}
  \hline
  攻撃力 & HP & コスト & 特殊効果 & 枚数 \\ \hline
  1 & 1 & 0 & 無し & 2 \\ \hline
  2 & 1 & 1 & 無し & 2 \\ \hline
  3 & 2 & 2 & 無し & 2 \\ \hline
  4 & 3 & 3 & 無し & 2 \\ \hline
  5 & 4 & 4 & 無し & 2 \\ \hline
  2 & 2 & 2 & 召喚 & 2 \\ \hline
  2 & 3 & 3 & 召喚 & 2 \\ \hline
  1 & 1 & 1 & 循環 & 2 \\ \hline
  1 & 3 & 2 & 循環 & 2 \\ \hline
  2 & 1 & 2 & 速攻 & 2 \\ \hline
  3 & 1 & 3 & 速攻 & 2 \\ \hline
  1 & 2 & 2 & 攻撃 & 2 \\ \hline
  2 & 3 & 3 & 攻撃 & 2 \\ \hline
  1 & 1 & 1 & 治癒 & 2 \\ \hline
  2 & 1 & 3 & 治癒 & 2 \\ \hline
  \end{tabular}
  \end{table}

また, ゲームバランス調整のためのシミュレーションのために学習したエージェントを用いる予定のため学習相手にはルールベースで作成した好戦的な行動ルーチンを組んだ.アルゴリズム \ref{alg1} に作成した対戦相手の行動ルーチンを示す.
\begin{figure}[ht]
  \begin{algorithm}[H]
      \caption{敵の行動}
      \label{alg1}
      \begin{algorithmic}[1] 
      \FOR{手札のカード}
      \IF{盤面にプレイできる}
      \STATE カードをプレイ
      \ELSE
      \STATE pass
      \ENDIF
      \ENDFOR
      \FOR{自盤面のカード}
      \IF{敵の盤面に 1 回の攻撃で倒せるカードがある}
      \STATE そのカードを選んで攻撃
      \ELSE
      \STATE 敵プレイヤーを攻撃
      \ENDIF
      \ENDFOR
      \end{algorithmic}
  \end{algorithm}
\end{figure}

また, 表 \ref{table:state} , \ref{table:action} に定義し直した状態空間, 行動空間を示す.

\begin{table}[h]
  \centering
  \caption{定義した状態空間}
  \label{table:state}
  \begin{tabular}{|c|c|c|c|}
  \hline
  状態説明                        & 次元数        & 最小値        & 最大値         \\ \hline
  自, 敵プレイヤーのHP  & 2 & 0 & 20 \\
  \hline
  自, 敵プレイヤーのコスト & 2 & 0 & 5 \\ \hline

  手札 1 $\sim$9 の HP と攻撃力      & 18         & 0          & 20          \\ \hline
  手札 1 $\sim$ 9 のコスト & 9 & 0 & 5 \\ \hline
  手札 1 $\sim$ 9 の特殊効果 &  9 & 0 & 5 \\ \hline
  自盤面 1 $\sim$ 5 の HP と攻撃力     & 10         & 0          & 20          \\ \hline
  敵盤面 1 $\sim$ 5 の HP と攻撃力     & 10         & 0          & 20          \\ \hline
  自盤面 1 $\sim$ 5 がターン中行動可能かどうか & 5          & 0          & 1           \\ \hline
  お互いのライブラリの残り枚数     & 2 & 0 & 15 \\ \hline
  \end{tabular}
  \end{table}

  \begin{table}[h]
    \centering
    \caption{定義した行動空間}
    \label{table:action}
    \begin{tabular}{|c|c|}
    \hline
    行動説明                          & 次元数        \\ \hline
    手札 1 $\sim$ 9 を自盤面に出す             & 9          \\ \hline
    手札 1 $\sim$ 9 を自盤面に出さない & 9 \\ \hline
    自盤面 1 が敵盤面 1 $\sim$ 5 に攻撃 or 何もしない or 敵プレイヤーに攻撃    & 7          \\ \hline
    自盤面 2 が敵盤面 1 $\sim$ 5 に攻撃 or 何もしない or 敵プレイヤーに攻撃    & 7   \\ \hline
    自盤面 3 が敵盤面 1 $\sim$ 5 に攻撃 or 何もしない or 敵プレイヤーに攻撃    & 7\\ \hline
    自盤面 4 が敵盤面 1 $\sim$ 5 に攻撃 or 何もしない or 敵プレイヤーに攻撃    & 7 \\ \hline
    自盤面 5 が敵盤面 1 $\sim$ 5 に攻撃 or 何もしない or 敵プレイヤーに攻撃    & 7\\ \hline
    \end{tabular}
    \end{table}
  
  報酬は以下のように定義した.
  \begin{equation*}
    \label{reward}
    reward = 0.0,
    \quad 
    \mathrm{1 エピソード終了後}
    reward \text{ = }
    \left\{
      \begin{aligned}
          1.0 \quad & (学習プレイヤーの勝利) \\
          -1.0 \quad & (敵プレイヤーの勝利) \\
      \end{aligned}
      \right.
  \end{equation*}

  \subsection{実験 1}
  先週は 先攻側で DQN を用いて 1000000 ステップ実験を行い, 10000 回ゲームを実行し算出された勝率は 0.1461 とかなり低い数値だった. そこで, ステップ数の問題かどうか調べるため学習ステップを 10 倍の 10000000 ステップに増やして実験した.

  \subsection{実験 1 結果}
  図 \ref{fig:DQN} に実験 1 における学習時の獲得報酬の平均の推移を示す.
  \begin{figure}[h]
    \centering
    \includegraphics[width=120mm]{assets/result1.eps}
    \caption{実験 1 における獲得報酬平均の推移}
    \label{fig:DQN}
  \end{figure}
  
  200 エピソードにおける平均獲得報酬が -0.9 付近で安定しており学習が進んでいないことが判明した.勝率は記録していませんでした.申し訳ありません.
  
\section{実験 2}
実験 1 の結果を受けてエージェントの行動空間に 2 種類変更を施して学習 $\rightarrow$ 勝率計算の流れで実験した.
また, 共通して DQN における $\mathrm{\epsilon-greedy}$ に改良を施した.
\begin{itemize}
  \item 変更前
  \begin{equation*}
    \epsilon = \mathrm{max}(\mathrm{\epsilon_{min}} \quad , \quad -\frac{\mathrm{\epsilon_{max}} - \mathrm{\epsilon_{min}}}{\mathrm{stepnum}} (stepcount) + \mathrm{\epsilon_{max}})
  \end{equation*}
  学習時に, $stepcount$ が $\mathrm{stepnum}$ ステップに達するまで$\mathrm{\epsilon_{max}}$ から $\mathrm{\epsilon_{min}}$ へと線形的に減少する.
  \item 変更後
  \begin{equation*}
    \epsilon = \mathrm{max}(\mathrm{\epsilon_{min}} \quad , \quad \mathrm{\epsilon_{min}} + (\mathrm{\epsilon_{max} - \epsilon_{min}}) \exp(- \frac{stepcount}{\mathrm{\epsilon_{decay}}}))
  \end{equation*}
  学習時に, $\mathrm{\epsilon_{decay}}$ に応じて, $\mathrm{\epsilon_{max}}$ から $\mathrm{\epsilon_{min}}$ へと指数的に減少する.
\end{itemize}
表 \ref{table:updateparam} にこの方策に基づいた実験 2 における DQN のパラメータを示す.

\begin{table}[h]
  \centering
  \caption{DQNのパラメータ}
  \label{table:updateparam}
  \begin{tabular}{|c||c|}
  \hline
  方策                 & ε-greedy \\ \hline
  $\mathrm{\epsilon_{max}}$                      & 1.0      \\ \hline
  $\mathrm{\epsilon_{min}}$                  & 0.05      \\ \hline
  $\mathrm{\epsilon_{decay}}$    & 学習ステップ数 / 10.0      \\ \hline       
  全結合層の活性化関数             & ReLU     \\ \hline
  全結合層の次元                & 64       \\ \hline
  最適化アルゴリズム              & Adam     \\ \hline
  Target Network 更新重み              & 0.5     \\ \hline
  Exprience Memory への書き込み開始step & 10000 \\ \hline
  Experience Replayのメモリ量 & 50000  \\ \hline
  \end{tabular}
  \end{table}

\subsection{実験 2 - 1 }
盤面にあるカードに対して「相手プレイヤーに攻撃」という選択肢があるにもかかわらず「何もしない」という選択肢を選ぶのは特にメリットがないと感じたため表 \ref{table:action2} のように行動空間を変更した.

\begin{table}[h]
  \centering
  \caption{実験 2 - 1 で定義した行動空間 (太字は変更した箇所)}
  \label{table:action2}
  \begin{tabular}{|c|c|}
  \hline
  行動説明                          & 次元数        \\ \hline
  手札 1 $\sim$ 9 を自盤面に出す             & 9          \\ \hline
  手札 1 $\sim$ 9 を自盤面に出さない & 9 \\ \hline
  \textbf{自盤面 1 が敵盤面 1 $\sim$ 5 に攻撃 or 敵プレイヤーに攻撃}    &  \textbf{6}          \\ \hline
  \textbf{自盤面 2 が敵盤面 1 $\sim$ 5 に攻撃 or 敵プレイヤーに攻撃}    & \textbf{6}   \\ \hline
  \textbf{自盤面 3 が敵盤面 1 $\sim$ 5 に攻撃 or 敵プレイヤーに攻撃}    & \textbf{6} \\ \hline
  \textbf{自盤面 4 が敵盤面 1 $\sim$ 5 に攻撃 or 敵プレイヤーに攻撃}    & \textbf{6} \\ \hline
  \textbf{自盤面 5 が敵盤面 1 $\sim$ 5 に攻撃 or 敵プレイヤーに攻撃}    & \textbf{6} \\ \hline
  \end{tabular}
  \end{table}

この条件下で先攻側を 3000000 ステップ学習し, 10000 回ゲームを実行し勝率を計算した.

\subsection{実験 2 - 1 結果}
図 \ref{fig:result2-1} に実験 2 - 1 における学習時の 500 エピソードの平均獲得報酬の推移を示す.
  \begin{figure}[h]
    \centering
    \includegraphics[width=120mm]{assets/result2.eps}
    \caption{実験 2 - 1 における平均獲得報酬の推移}
    \label{fig:result2-1}
  \end{figure}

また, 表 \ref{table:result2-1} に勝率を示す.

\begin{table}[t]
  \centering
  \caption{実験 2 - 1 結果}
  \vspace{-0.3cm}
  \label{table:result2-1}
  \scalebox{1.0}[1.0]{
    \begin{tabular}{|c|c|}
      \hline
      手法 & 勝率 \\ \hline
      DQN & 0.6022 \\ \hline
      対戦相手と同じ戦略 & \textbf{0.6425} \\ \hline

      \end{tabular}
  }
  \end{table}

実験 1 に比べ , 学習は進んでいくにつれ平均の報酬が高くなり, 約 6 割の勝率を記録した.
しかし, 先攻側にアルゴリズム \ref{alg1} で示した行動ルーチンを持つプレイヤーを配置した場合の勝敗に比べると小さく, ルールベースな敵 AI よりも強化学習を用いる意味がない.

\subsection{実験 2 - 2 }
手札において「盤面に出さない」という選択肢が学習が進まなくなる要因であると当たりをつけ, プレイヤーの選択肢に「ターンエンド」を追加した. 表 \ref{table:action2-2} に定義した行動空間を示す.

\begin{table}[h]
  \centering
  \caption{実験 2 - 2 で定義した行動空間 (太字は変更した箇所)}
  \label{table:action2-2}
  \begin{tabular}{|c|c|}
  \hline
  行動説明                          & 次元数        \\ \hline
  手札 1 $\sim$ 9 を自盤面に出す             & 9          \\ \hline
  自盤面 1 が敵盤面 1 $\sim$ 5 に攻撃 or 敵プレイヤーに攻撃    & 6          \\ \hline
  自盤面 2 が敵盤面 1 $\sim$ 5 に攻撃 or 敵プレイヤーに攻撃    & 6   \\ \hline
  自盤面 3 が敵盤面 1 $\sim$ 5 に攻撃 or 敵プレイヤーに攻撃    & 6\\ \hline
  自盤面 4 が敵盤面 1 $\sim$ 5 に攻撃 or 敵プレイヤーに攻撃    & 6 \\ \hline
  自盤面 5 が敵盤面 1 $\sim$ 5 に攻撃 or 敵プレイヤーに攻撃    & 6\\ \hline
  \textbf{ターンエンド} & \textbf{1} \\ \hline
  \end{tabular}
  \end{table}

  この条件下で先攻側を 3000000 ステップ学習し, 10000 回ゲームを実行し勝率を計算した.

\subsection{実験 2 - 2 結果}
図 \ref{fig:result2-2} に実験 2 - 2 における学習時の 500 エピソードの平均獲得報酬の推移を示す.
  \begin{figure}[h]
    \centering
    \includegraphics[width=120mm]{assets/result3.eps}
    \caption{実験 2 - 2 における平均獲得報酬の推移}
    \label{fig:result2-2}
  \end{figure}

また, 表 \ref{table:result2-2} に勝率を示す.

\begin{table}[t]
  \centering
  \caption{実験 2 - 2 結果}
  \vspace{-0.3cm}
  \label{table:result2-2}
  \scalebox{1.0}[1.0]{
    \begin{tabular}{|c|c|}
      \hline
      手法 & 勝率 \\ \hline
      DQN & \textbf{0.9708} \\ \hline
      対戦相手と同じ戦略 & 0.6425 \\ \hline

      \end{tabular}
  }
  \end{table}
学習時の獲得平均報酬が大きく上昇し, ルールベースで作成した敵に対して 9 割 5 分以上の勝率を記録することができた.
学習したエージェントの行動を見てみると先攻プレイヤーらしく, 積極的に相手プレイヤーに攻撃し, 手札からも攻撃の特殊効果を持つカードを優先的にプレイしていた.

\section{後攻側の学習}
高い勝率を記録した実験 2 - 2 の条件で学習プレイヤーを後攻に配置して, 1000000 ステップ学習後 10000 回ゲームを実行して勝率を計算した.
図 \ref{fig:result4} に学習時の 500 エピソードの平均獲得報酬の推移を示す.

\begin{figure}[h]
  \centering
  \includegraphics[width=120mm]{assets/result4.eps}
  \caption{学習時における平均獲得報酬の推移}
  \label{fig:result4}
\end{figure}

また, 表 \ref{table:result4} に勝率を示す.

\begin{table}[t]
\centering
\caption{実験結果}
\vspace{-0.3cm}
\label{table:result4}
\scalebox{1.0}[1.0]{
  \begin{tabular}{|c|c|}
    \hline
    手法 & 勝率 \\ \hline
    DQN & \textbf{0.7969} \\ \hline
    対戦相手と同じ戦略 & 0.3575 \\ \hline

    \end{tabular}
}
\end{table}
後攻側の学習においてもルールベースで作成した敵に対して高い勝率を記録した.

\section{今後の課題}

\begin{quote}
  \begin{itemize}
   \item 対戦相手の行動の改善
   \par
   今回の実験で, エージェントの行動空間の定義を改善することができ学習によりルールベースで作成した敵よりも高い勝率を残すエージェントを作成することができた.
   しかし現在は学習の際,対戦相手は好戦的な行動ルーチンに基づいて行動している. バランス調整のシミュレーション回す際に学習したエージェントを用いる予定なので防戦的な行動ルーチンを作成し, エピソードごとに敵のルーチンを変えるなどしてどちらにも勝てるようなエージェントを作成する必要がある.
  \end{itemize}
 \end{quote}

\end{document}